\documentclass[]{scrbook}
\usepackage[T1]{fontenc}    % avoid garbled Unicode text in pdf
\usepackage[utf8]{inputenc} % use utf8 file encoding for TeX sources
\usepackage[german]{babel}  % german hyphenation, quotes, etc
\usepackage{hyperref}       % detailed hyperlink/pdf configuration
\hypersetup{                % ‘texdoc hyperref‘ for options
pdftitle={PSE: Entwicklung eines relationalen Debuggers - Pflichtenheft},%
,%
}
\usepackage{graphicx}       % provides commands for including figures
\usepackage{csquotes}       % provides \enquote{} macro for "quotes"
\usepackage{enumitem}


\title{PSE: Entwicklung eines relationalen Debuggers - Pflichtenheft}
\author{Benedikt Wagner, Chiara Staudenmaier, Étienne ?, Joana Plewnia, Pascal Zwick, Ulla Scheler}

\begin{document}

\maketitle

\section{Produktübersicht}
%kurze Übersicht über das Produkt

\section{Zielbestimmung}
\subsection{Musskriterien}
\subsection{Sollkriterien}
\subsection{Kannkriterien}
\subsection{Abgrenzungskriterien}

\section{Produkteinsatz}
%Anwendungsbereiche, Zielgruppen, Betriebsbedingungen

\section{Funktionale Anforderungen}
\begin{itemize}[nosep]
\item[FA10] %genaue und detaillierte Beschreibung der einzelnen Produktfunktionen
\end{itemize}

\section{Produktdaten}
\begin{itemize}[nosep]
\item[PD10] %langfristig zu speichernde Daten aus Benutzersicht
\end{itemize}

\section{Nichtfunktionale Anforderungen}
\begin{itemize}[nosep]
\item[NF10] %Anforderungen bezüglich Ziel und Genauigkeit
\end{itemize}

\section{Weitere nichtfunktionale Anforderungen}
\begin{itemize}
\item[WNF10] %einzuhaltende Gesetze, Normen, Urheber-und Markenrechte, Sicherheitsanforderungen, Plattformabhängigkeiten
\end{itemize}

\section{Qualitätsanforderungen}

\section{Globale Testfälle und Szenarien}
\subsection{Anwendungsfälle}
\subsection{Testfälle}
\subsection{Testszenarien}

\section{Systemmodelle}
%Architektur, Verhalten, usw

\section{Benutzungsoberfläche}
%Gui-Skizzen, Erklärungen der Menüs, usw

\section{Spezielle Anforderungen an die Entwicklungsumgebung}

\section{Zeit- und Ressourcenplanung}

\section{Ergänzungen}

\section{Glossar}




\end{document}

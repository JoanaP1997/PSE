\documentclass[parskip=full]{scrartcl}
\usepackage[T1]{fontenc}    % avoid garbled Unicode text in pdf
\usepackage[utf8]{inputenc} % use utf8 file encoding for TeX sources
\usepackage[german]{babel}  % german hyphenation, quotes, etc
\usepackage{hyperref}       % detailed hyperlink/pdf configuration
\hypersetup{                % ‘texdoc hyperref‘ for options
pdftitle={PSE: Entwicklung eines relationalen Debuggers - Testbericht},%
,%
}
\usepackage{graphicx}       % provides commands for including figures
\usepackage{csquotes}       % provides \enquote{} macro for "quotes"
\usepackage[nonumberlist]{glossaries}     % provides glossary commands
\usepackage{enumitem}
\usepackage{xcolor}
\newcommand\frage[1]{\textcolor{red}{#1}}
\renewcommand{\glstextformat}[1]{\textbf{\color{blue}\em #1}}

\font\myfont=cmr12 at 20pt

\title{
	\vspace{2cm}
	\myfont 
	Praxis der Softwareentwicklung:\\ 
	Entwicklung eines relationalen Debuggers\\
}
\subtitle{
	\vspace{1cm}
	\myfont
	Testbericht
}

\author{
	\vspace{1cm} \\
	Benedikt Wagner\\
	\and 
        \vspace{1cm} \\ 
        Chiara Staudenmaier\\
        \and 
        \vspace{1cm} \\
        Etienne Brunner\\
	\and Joana Plewnia\\
	\and Pascal Zwick\\
	\and Ulla Scheler\\
	\vspace{1cm}
	\and Betreuer: Mihai Herda, Michael Kirsten
	\vspace{4cm}
}

\begin{document}
\clearpage
\maketitle
\pagenumbering{gobble}
\newpage

\tableofcontents
\newpage
\pagenumbering{arabic}

\section{Einleitung}

\begin{figure}[!h]
\centering
\includegraphics[width=0.6\textwidth]{../Plichtenheft/logo_nongi.png}
\caption{Produktlogo}
\end{figure}

\section{Komponententest}
\subsection{Funktionale Tests}
\subsubsection{Interpreter}
Die Aufgabe des Interpreters ist das Ausführen der vom Nutzer als Quelltext gegebenen Programme. 
Da Commandklassen die Funktionsfähigkeit von Termen vorraussetzen und diese ein korrekt implementiertes Typsystem erfordern wurde dies hier in der Reihenfolge des Schreibens der Unittests berücksichtigt. 
So wurden zunächst Unittest für die Implementierungen der abstrakten Klasse \textit{TermValue} geschrieben.
\subsection{Belastungstests}

\section{Integrationstest}

\subsection{Funktionale Tests}

\subsection{Belastungstests}

\section{UserInterface Test}

\subsection{Testplan / Übersicht}
-Monkey Testing \\
-Testbereiche: Visual Design Functionality Performance Security Usability Compliance

\subsection{Monkey Testing}
\subsubsection{Dummes Monkey Testing}
\subsubsection{Intelligentes Monkey Testing}

\subsection{Testfälle}

\end{document}

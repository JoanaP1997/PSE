\documentclass[parskip=full]{scrartcl}
\usepackage[T1]{fontenc}    % avoid garbled Unicode text in pdf
\usepackage[utf8]{inputenc} % use utf8 file encoding for TeX sources
\usepackage[german]{babel}  % german hyphenation, quotes, etc
\usepackage{hyperref}       % detailed hyperlink/pdf configuration
\hypersetup{                % ‘texdoc hyperref‘ for options
pdftitle={PSE: Entwicklung eines relationalen Debuggers - Implementierungsdokument},%
,%
}
\usepackage{graphicx}       % provides commands for including figures
\usepackage{csquotes}       % provides \enquote{} macro for "quotes"
\usepackage[nonumberlist]{glossaries}     % provides glossary commands
\usepackage{enumitem}
\usepackage{xcolor}
\usepackage{lscape}
\newcommand\frage[1]{\textcolor{red}{#1}}


\font\myfont=cmr12 at 16pt

\title{
	\vspace{2cm}
	\myfont 
	Praxis der Softwareentwicklung:\\ 
	Entwicklung eines relationalen Debuggers\\
}
\subtitle{
	\vspace{1cm}
	\myfont
	Implementierungsdokument
}
\author{
	\vspace{1cm} \\
	Benedikt Wagner\\
	\texttt{udpto@student.kit.edu}
	\and \vspace{1cm} \\ Chiara Staudenmaier\\
	\texttt{uzhtd@student.kit.edu}
	\and Etienne Brunner\\
	\texttt{urmlp@student.kit.edu}
	\and Joana Plewnia\\
	\texttt{uhfpm@student.kit.edu} 
	\and Pascal Zwick\\
	\texttt{uyqpk@student.kit.edu}
	\and Ulla Scheler\\
	\texttt{ujuhe@student.kit.edu}
	\vspace{1cm}
	\and Betreuer: Mihai Herda, Michael Kirsten
	\vspace{4cm}
}


\begin{document}
\clearpage
\maketitle
\pagenumbering{gobble}
\newpage

\tableofcontents
\newpage
\pagenumbering{arabic}

\section{Einleitung}

\section{Zeitablauf}

\section{Umsetzung der nichtfunktionalen-Anforderungen}

\section{Umsetzung von Entwurfsentscheidungen}
\subsection{Model-View-Control}

\subsection{Singleton}
z.B. Command Panel
\subsection{Kompositum}
z.B. while-Command
\subsection{Fassade}
z.B. Control-Facade
\subsection{Strategie}
z.B. FileWriter
\subsection{Beobachter}
z.B. Gui-Facade

\section{Änderungen zum Entwurf}

\subsection{UserInterface}
\subsection{Control}
\subsection{FileHandler}
+ FileHandlerFacade: savePropertiesFile()
\subsection{DebugLogic}
\subsubsection{Debugger}

+ DebugLogicFacade: extends Observable statt Subject
* DebugLogicFacade: notifyObservers(Object arg); suggestStepSize void statt String,

+ DebugControl: setStepSize(programID, stepSize); reset();
\subsubsection{Interpreter}
\paragraph{TraceIterator}
Ursprünglich sollte der TraceIterator Umsetzung des TraceIterators mit einer gesonderten Klasse, die das Java-Interface Iterator implementiert \\
Änderung: Der Iterator über den Trace wurde mit Hilfe eines von Java bereitgestellen \textit{ListIterators} implementiert, den die Klasse \textit{Trace} mit ihrer \textit{iterator()}-Methode zurückgibt.
* TermValue: abstract class statt interface \\
* Listiterator: vereinfacht

\section{Anhang}

\end{document}